
%%%%%%%%%%%%%%%%%%%%%%%%%%%%%%%%%%%%%%%%%%%%%%%%%%%%%%%%%%%%%%%%%%%%%%%%%%%%%
%%%%% C++ and Memory %%%%%%%%%%%%%%%%%%%%%%%%%%%%%%%%%%%%%%%%%%%%%%%%%%%%%%%%
%%%%%%%%%%%%%%%%%%%%%%%%%%%%%%%%%%%%%%%%%%%%%%%%%%%%%%%%%%%%%%%%%%%%%%%%%%%%%
\begin{frame}
\pointedsl{
	Stack / Heap
}
\end{frame}

% See http://stackoverflow.com/questions/79923/what-and-where-are-the-stack-and-heap
% See http://gribblelab.org/CBootcamp/7_Memory_Stack_vs_Heap.html
% See http://www.learncpp.com/cpp-tutorial/79-the-stack-and-the-heap/

%
% Content:
% 1. Stack -vs- heap
% 2. Stack: variable scope (when is the destructor called?)
% 3. Heap: new & delete (and memory leaks)
% 4. Heap: arrays with new and delete
% 5. Smart pointers

%%%%%%%%%%%%%%%%%%%%%%%%%%%%%%%%%%%%%%%%%%%%%%%%%%%%%%%%%%%%%%%%%%%%%%%%%%%%%
\begin{frame}[fragile]
\frametitle{Stack -vs- heap}
\misc{
	The \emph{stack} is a place where variables live and die depending on their location in the program.
	
	The \emph{heap} is the place where variables live (and die) independently of their location in the program.
}
\end{frame}

%%%%%%%%%%%%%%%%%%%%%%%%%%%%%%%%%%%%%%%%%%%%%%%%%%%%%%%%%%%%%%%%%%%%%%%%%%%%%

\begin{frame}[fragile]
\frametitle{Stack: life and death of a variable}
\begin{lstlisting}
int life_of_a_variable(bool b){ // b born
  int a = 0; // a born
  if(b){
    int c = 10; // c born
    ... // do other things
  } // c dead
  return a; // a copied, then dead
} // b dead
\end{lstlisting}
\misc{
	On the \emph{stack}, local variables die as soon as the program reaches back a level where the considered variables were not defined.
	
	They die with the call of their \emph{destructor}.
}
\end{frame}

%%%%%%%%%%%%%%%%%%%%%%%%%%%%%%%%%%%%%%%%%%%%%%%%%%%%%%%%%%%%%%%%%%%%%%%%%%%%%

\begin{frame}[fragile]
\frametitle{Using the stack}
\begin{center}
\B
\begin{tabular}{ll}
\B \textbf{Pro} & \B \textbf{Contra} \\\hline
\B Fast access & \B \ctext{return} requires a copy \\
\B Managed life-cycle & \B Limited memory
\end{tabular}
\end{center}
\misc{
	The stack should be reserved for \textbf{simple} and \textbf{local} variables.
	
	\textbf{Heavy} or \textbf{complex} variables should be allocated on the heap.
}
\end{frame}

%%%%%%%%%%%%%%%%%%%%%%%%%%%%%%%%%%%%%%%%%%%%%%%%%%%%%%%%%%%%%%%%%%%%%%%%%%%%%
\begin{frame}[fragile]
\lstset{ numbers=none }
\frametitle{Heap: \ctext{new} and \ctext{delete}}
\begin{lstlisting}
// forward-declaration
class User; // with complex data

User *myself = new User("Alex");
// use myself, possibly return
//     from a function
// eventually delete:
delete myself;
\end{lstlisting}
\misc{
	Using \ctext{new} enforces that the variable is created on the heap.
	
	Instances created on the heap are not deleted automatically, the programmer must use \ctext{delete} on these instances to free the memory.
}
\end{frame}

\begin{frame}[fragile]
\lstset{ numbers=none }
\frametitle{Heap: \ctext{new} and \ctext{delete}}
\begin{lstlisting}
struct Image {
  int x, y;
  ...
  void resize(Size newSize);
};

Image *img = new Image;
img->x = 10; // same as (*img).x
img->y = 10; // same as (*img).y
img->resize(100, 100);
\end{lstlisting}
\misc{
	Instances on the heap are dealt with using pointers.
	
	Accessing their members is done using the \ctext{->} operator.
}
\end{frame}

%%%%%%%%%%%%%%%%%%%%%%%%%%%%%%%%%%%%%%%%%%%%%%%%%%%%%%%%%%%%%%%%%%%%%%%%%%%%%
\begin{frame}[fragile]
\frametitle{Heap: arrays with \ctext{new[]} and \ctext{delete[]}}
\begin{lstlisting}
// fibonacci sequence
int *fibSequence(int N){
  int *seq = new int[N];
  seq[0] = seq[1] = 1;
  for(int i = 2; i < N; ++i)
    seq[i] = seq[i-1] + seq[i-2];
  return seq;
}
\end{lstlisting}
\misc{
	Creating arrays on the heap (especially any array of variable size) is done using
	the \ctext{new[]} operator.
	
	Freeing uses the corresponding \ctext{delete[]} operator.
}
\end{frame}

%%%%%%%%%%%%%%%%%%%%%%%%%%%%%%%%%%%%%%%%%%%%%%%%%%%%%%%%%%%%%%%%%%%%%%%%%%%%%
\begin{frame}[fragile]
\frametitle{Smart pointers}

\misc{
	The \emph{stack} is a place where variables live and die depending on their location in the program.
	
	The \emph{heap} is the place where variables live (and die) independently of their location in the program.
}
\end{frame}