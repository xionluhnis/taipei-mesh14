
%%%%%%%%%%%%%%%%%%%%%%%%%%%%%%%%%%%%%%%%%%%%%%%%%%%%%%%%%%%%%%%%%%%%%%%%%%%%%
%%%%% Introduction %%%%%%%%%%%%%%%%%%%%%%%%%%%%%%%%%%%%%%%%%%%%%%%%%%%%%%%%%%
%%%%%%%%%%%%%%%%%%%%%%%%%%%%%%%%%%%%%%%%%%%%%%%%%%%%%%%%%%%%%%%%%%%%%%%%%%%%%

%
% Content:
% 1. OpenMesh
% 2. Open Geometry Processing
% 3. Polygonal Meshes
%

%%%%%%%%%%%%%%%%%%%%%%%%%%%%%%%%%%%%%%%%%%%%%%%%%%%%%%%%%%%%%%%%%%%%%%%%%%%%%
\fbckg{backgrounds/3dparis2}
\begin{frame}
\pointedsl{OpenMesh}
\end{frame}

\setbeamercolor{frametitle}{fg=white}

\begin{frame}
\frametitle{OpenMesh}
\itemized{
	\item C++ library to work with polygonal meshes
	\item Half-edge data structure
	\item Efficient representation and manipulation
	\item Developed at the \href{http://www.graphics.rwth-aachen.de/}{Computer Graphics Group, RWTH Aachen}
	\item Led by Prof. Leif Kobbelt
}
\end{frame}

% based on http://www.openmesh.org/media/Documentations/OpenMesh-Doc-Latest/a00012.html
% more at http://opengp.github.io/tutorial.html

%%%%%%%%%%%%%%%%%%%%%%%%%%%%%%%%%%%%%%%%%%%%%%%%%%%%%%%%%%%%%%%%%%%%%%%%%%%%%
\setbeamercolor{frametitle}{fg=black}
\fbckg{backgrounds/3dparis-gray-edges}
\begin{frame}
\frametitle{OpenGP}
\framedsl{
	Open Geometry Processing?
}
\end{frame}

\begin{frame}
\frametitle{OpenGP}
\itemized{
	\item C++ library similar to OpenMesh
	\item Lightweight, simpler to use
	\item Developed at the \href{http://graphics.uni-bielefeld.de/}{Bielefeld Graphics \& Geometry Group}
	\item Led by Prof. Mario Botsch
}
\end{frame}

%%%%%%%%%%%%%%%%%%%%%%%%%%%%%%%%%%%%%%%%%%%%%%%%%%%%%%%%%%%%%%%%%%%%%%%%%%%%%
\fbckg{backgrounds/platonic}
\begin{frame}
\framedsl{
	Polygonal Meshes
}
\end{frame}

\begin{frame}
\frametitle{Polygons}
\itemized{
	\item Work with any $N$-gone faces
	\item Here focus on triangles
	\item Any polygon can be \emph{triangulated}
}
\end{frame}