
%%%%%%%%%%%%%%%%%%%%%%%%%%%%%%%%%%%%%%%%%%%%%%%%%%%%%%%%%%%%%%%%%%%%%%%%%%%%%
%%%%% Templating %%%%%%%%%%%%%%%%%%%%%%%%%%%%%%%%%%%%%%%%%%%%%%%%%%%%%%%%%%%%
%%%%%%%%%%%%%%%%%%%%%%%%%%%%%%%%%%%%%%%%%%%%%%%%%%%%%%%%%%%%%%%%%%%%%%%%%%%%%
\begin{frame}
\pointedsl{
	Classes
}
\end{frame}

% TODO
% Content:
% 1. Basic class usage
% 2. Class variables & method declaration
% 3. Class definition
% 4. Constructor (default, copy, other), destructor
% 5. Accessors (public, protected, private)
% 6. Instance and static members
% 7. Operators and overloading

%%%%%%%%%%%%%%%%%%%%%%%%%%%%%%%%%%%%%%%%%%%%%%%%%%%%%%%%%%%%%%%%%%%%%%%%%%%%%
\begin{frame}[fragile]
\frametitle{Basic class usage}
\begin{lstlisting}
struct point {
  int x, y;
}; // don't forget ;

point p1, p2;
p1.x = 1;
p1.y = 2;
p2 = p1; // copy
p2.y = 2;
\end{lstlisting}
\misc{
	Classes (keyword \ctext{class}) and structures (keyword \ctext{struct}) describe custom types made of properties (here \ctext{x} and \ctext{y}).
}
\end{frame}

%%%%%%%%%%%%%%%%%%%%%%%%%%%%%%%%%%%%%%%%%%%%%%%%%%%%%%%%%%%%%%%%%%%%%%%%%%%%%
\begin{frame}[fragile]
\frametitle{Class member declaration}
\begin{lstlisting}
// declaration
struct vec {
    float val[2]; // variable
    // instance method:
    float sqLength() const;
    // static method:
    static vec constant(float a);
};
\end{lstlisting}
\misc{
	They can have specific methods (\ctext{sqLength}, \ctext{constant}).
}
\end{frame}

%%%%%%%%%%%%%%%%%%%%%%%%%%%%%%%%%%%%%%%%%%%%%%%%%%%%%%%%%%%%%%%%%%%%%%%%%%%%%
\begin{frame}[fragile]
\frametitle{Class definition}
\begin{lstlisting}
// definition
float vec::sqLength() {
    float sum = 0;
    for(int i = 0; i < dim; ++i)
        sum += val[i] * val[i];
    return sum;
}
\end{lstlisting}
\misc{
	The methods can be defined outside of the declaration.
	
	The usual pattern is:
	\begin{itemize}
		\item declare class in \ctext{classname.h}
		\item define class in \ctext{classname.cpp}
	\end{itemize}
}
\end{frame}

%%%%%%%%%%%%%%%%%%%%%%%%%%%%%%%%%%%%%%%%%%%%%%%%%%%%%%%%%%%%%%%%%%%%%%%%%%%%%
\begin{frame}[fragile]
\frametitle{Constructors and destructor}
\misc{
	Some methods are special, including
	\begin{itemize}
		\item the \textbf{default constructor} which is called for instances without specific initialization
		\item the \textbf{copy constructor} which is used for passing argument by value and when assigning a copy of a variable to a new variable
		\item the \textbf{destructor} which is called when the instance is deleted
	\end{itemize}
}
\end{frame}

%%%%%%%%%%%%%%%%%%%%%%%%%%%%%%%%%%%%%%%%%%%%%%%%%%%%%%%%%%%%%%%%%%%%%%%%%%%%%
\begin{frame}[fragile]
\frametitle{Instance creation}
\begin{lstlisting}
struct point {
  int x, y;

  point(int a, int b); // my constructor
  point(); // default constructor
  point(const point &p); // copy constructor
  ~point(); // destructor
};

point a(1, 2); // my constructor
point b; // default constructor
point c(a); // copy constructor
// assignment using copy constructor
point d = e;
\end{lstlisting}
\end{frame}

%%%%%%%%%%%%%%%%%%%%%%%%%%%%%%%%%%%%%%%%%%%%%%%%%%%%%%%%%%%%%%%%%%%%%%%%%%%%%
\begin{frame}[fragile]
\frametitle{Accessors}
\misc{
	Members of classes can be 
	\begin{itemize}
		\item \ctext{public}: can be accessed from everywhere, 
		\item \ctext{protected}: can only be accessed by classes that extend it, or 
		\item \ctext{private}: can only be accessed by methods of that instance
	\end{itemize}
	
	By default, \ctext{struct}'s have every member \ctext{public} whereas \ctext{class}'es use the default accessor \ctext{private}.
}
\end{frame}

%%%%%%%%%%%%%%%%%%%%%%%%%%%%%%%%%%%%%%%%%%%%%%%%%%%%%%%%%%%%%%%%%%%%%%%%%%%%%
\begin{frame}[fragile]
\frametitle{Accessors}
\lstset{
  xrightmargin=0cm
}
\begin{columns}[c]
  \begin{column}{0.5\textwidth}
\begin{lstlisting}
struct point {
  int x, y;
  void print();
private:
  void debug();
};

// ok:
point a;
a.print();
// error:
a.debug();
\end{lstlisting}
  \end{column}
  \begin{column}{0.5\textwidth}
\lstset{ numbers=none, xleftmargin=0cm }
\begin{lstlisting}
class vec {
  int val[2];
public:
  vec(int a, int b);
  int sqLength();
};

// ok:
vec b(1, 2);
int l = b.sqLength();
// error:
b.val[0] = 1;
\end{lstlisting}
  \end{column}
\end{columns}
\end{frame}

%%%%%%%%%%%%%%%%%%%%%%%%%%%%%%%%%%%%%%%%%%%%%%%%%%%%%%%%%%%%%%%%%%%%%%%%%%%%%
\begin{frame}[fragile]
\frametitle{Static members}
\misc{
	Normal variable and methods are relative to the instance of the class that stores them.
	
	The keyword \ctext{static} can be used to declare variables or methods that are relative to the namespace of the class, shared by all instances.
}
\end{frame}

%%%%%%%%%%%%%%%%%%%%%%%%%%%%%%%%%%%%%%%%%%%%%%%%%%%%%%%%%%%%%%%%%%%%%%%%%%%%%
\begin{frame}[fragile]
\frametitle{Static members}
\begin{lstlisting}
struct id {
  const int value;
  id() : value(num++){}
  int count() { return num; }
private:
  static int num;
}
int id::num = 0;

id a, b;
// a.value == 0, b.value == 1
// a.count() == b.count() == 2
\end{lstlisting}
\end{frame}

%%%%%%%%%%%%%%%%%%%%%%%%%%%%%%%%%%%%%%%%%%%%%%%%%%%%%%%%%%%%%%%%%%%%%%%%%%%%%
\begin{frame}[fragile]
\frametitle{Classes (2)}
\begin{lstlisting}
// definitions
template <typename T, int dim>
T vec<T, dim>::sqLength() {
    T sum = 0;
    for(int i = 0; i < dim; ++i)
        sum += val[i] * val[i];
    return sum;
}
\end{lstlisting}
\misc{
	Member definitions can be done separately.
	
	Typically, the declaration would be in a \ctext{.h} file and the definition in a \ctext{.cpp} file.
}
\end{frame}

\begin{frame}[fragile]
\frametitle{Classes (3)}
\begin{lstlisting}
// type alias
typedef vec<float, 2> vec2f;

// class usage
vec2f a = vec2f::constant(2); // (2, 2)
float d = a.sqLength(); // d=8
\end{lstlisting}
\misc{
	Static properties and methods are called with the namespace operator (\ctext{::}) whereas
	instance members are accessed using the dot operator.
}
\end{frame}

\begin{frame}[fragile]
\frametitle{Operator overloading}
\begin{lstlisting}
vec2f operator +(const vec2f &v1,
                 const vec2f &v2){
    vec2f x;
    for(int i = 0; i < 2; ++i)
        x.val[i] = v1.val[i] + v2.val[i];
    return x;
}

// usage
vec2f a, b;
// hidden: initialization of a and b
vec2f c = a + b;
\end{lstlisting}
\end{frame}
