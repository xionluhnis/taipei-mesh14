
%%%%%%%%%%%%%%%%%%%%%%%%%%%%%%%%%%%%%%%%%%%%%%%%%%%%%%%%%%%%%%%%%%%%%%%%%%%%%
%%%%% Traversing a mesh %%%%%%%%%%%%%%%%%%%%%%%%%%%%%%%%%%%%%%%%%%%%%%%%%%%%%
%%%%%%%%%%%%%%%%%%%%%%%%%%%%%%%%%%%%%%%%%%%%%%%%%%%%%%%%%%%%%%%%%%%%%%%%%%%%%
\begin{frame}
\pointedsl{
	Topology
}
\end{frame}

%
% Content:
% 1. Adding vertices and faces
% 2. Deleting elements (vertices, edges, faces)
% 3. Other high-level operations (edge flip, edge/face split, halfedge collapse)

%%%%%%%%%%%%%%%%%%%%%%%%%%%%%%%%%%%%%%%%%%%%%%%%%%%%%%%%%%%%%%%%%%%%%%%%%%%%%
\begin{frame}[fragile]
\frametitle{Creating a mesh}
\misc{
	Meshes can be created from scratch by \textbf{adding} the geometry (mostly \textbf{vertices} and \textbf{faces}, opengp takes care of the edges).
}
\begin{lstlisting}
typedef Surface_mesh::Vertex Vertex;
Vertex v0, v1, v2, v3;
// add vertices (at given locations)
v0 = mesh.add_vertex(Point(0, 0, 0));
v1 = mesh.add_vertex(Point(1, 0, 0));
v2 = mesh.add_vertex(Point(0, 1, 0));
// add face
mesh.add_triangle(v0, v1, v2);
\end{lstlisting}
\end{frame}

\begin{frame}[fragile]
\frametitle{Creating a mesh}
\begin{lstlisting}[firstnumber=9]
// add quad
v3 = mesh.add_vertex(Point(-1, 1, 0));
mesh.add_quad(v0, v1, v2, v3);

// N-gone (perfect here)
int N = 20;
std::vector<Vertex> vert(N);
for(int i = 0; i < N; ++i) {
  Vertex v = mesh.add_vertex(
    Point(std::cos(2.0f * M_PI * i),
      0.0f, std::sin(2.0f * M_PI * i)) 
  );
  vert.push_back(v); // push at end
}
mesh.add_face(vert);
\end{lstlisting}
\end{frame}

%%%%%%%%%%%%%%%%%%%%%%%%%%%%%%%%%%%%%%%%%%%%%%%%%%%%%%%%%%%%%%%%%%%%%%%%%%%%%
\begin{frame}
\frametitle{Destroyig a mesh}
\misc{
	Elements can be deleted too using one of
	\begin{itemize}
		\item \ctext{mesh.delete\_vertex(v)}
		\item \ctext{mesh.delete\_edge(e)}
		\item \ctext{mesh.delete\_face(f)}
	\end{itemize}
	
	The other parts that depend on the deleted element are also removed!
}
\end{frame}

\begin{frame}
\framedsl{
	Beware when topology changes!
}
\end{frame}

\begin{frame}
\frametitle{Garbage collection}
\misc{
	Adding elements is safe.
	
	\phantom{x}
	
	Deleting elements can be dangerous. Elements are not directly
	removed, instead they are just \textbf{marked} as deleted.
	\\(but they are still in the data structure!)
	
	\phantom{x}
	
	Use \ctext{mesh.garbage\_collection()} to remove these elements and clean the mesh data!
}
\end{frame}

\begin{frame}
\frametitle{Garbage collection}
\misc{
	When iterating over elements, do not call \ctext{garbage\_collection()}!
	
	\phantom{x}
	
	Instead, check whether the element \ctext{x} you are working with has been deleted
	using \ctext{mesh.is\_deleted(x)}.
}
\end{frame}

%%%%%%%%%%%%%%%%%%%%%%%%%%%%%%%%%%%%%%%%%%%%%%%%%%%%%%%%%%%%%%%%%%%%%%%%%%%%%
\begin{frame}
\framedsl{
	Going farther
}
\end{frame}

\fbckg{backgrounds/topology-changes}
\begin{frame}
\frametitle{High-level topological operations}
\misc{
	The remeshing session will be about advanced topological operations such as \textbf{edge collapse}, \textbf{edge / face split}, \textbf{edge flip} and how to use them to improve the quality of our meshes.
}
\end{frame}