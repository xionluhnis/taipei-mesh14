
%%%%%%%%%%%%%%%%%%%%%%%%%%%%%%%%%%%%%%%%%%%%%%%%%%%%%%%%%%%%%%%%%%%%%%%%%%%%%
%%%%% Traversing a mesh %%%%%%%%%%%%%%%%%%%%%%%%%%%%%%%%%%%%%%%%%%%%%%%%%%%%%
%%%%%%%%%%%%%%%%%%%%%%%%%%%%%%%%%%%%%%%%%%%%%%%%%%%%%%%%%%%%%%%%%%%%%%%%%%%%%
\begin{frame}
\pointedsl{
	Hello world
}
\end{frame}


%%%%%%%%%%%%%%%%%%%%%%%%%%%%%%%%%%%%%%%%%%%%%%%%%%%%%%%%%%%%%%%%%%%%%%%%%%%%%
\begin{frame}[fragile]
\frametitle{IO code}
\begin{lstlisting}
 // create empty mesh
Surface_mesh mesh;

// load .obj / .stl / .off
mesh.read("myfile.obj");

// process the mesh
// ...

// write the mesh to a file
mesh.write("myfile_processed.obj");
\end{lstlisting}
\end{frame}

\begin{frame}
\frametitle{Processing}
\misc{
	Reading and writing meshes is simple.
	
	Processing them is also simple. We will cover
	\begin{enumerate}
		\item Methods to access connected elements\\(i.e. edges, faces, vertices and their neighborhood)
		\item Ways to traverse a mesh (all faces, all boundary edges, all vertices around one vertex, etc.)
		\item How to store data on the mesh elements
		\item Topological changes (adding, deleting elements)
	\end{enumerate}
}
\end{frame}