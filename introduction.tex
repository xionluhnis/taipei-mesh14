\documentclass{fancyslides} 
\usepackage[utf8]{inputenc}
\usepackage{times}
\usepackage{algorithm2e}
\usepackage{listings}


%%% Beamer settings (do not change)
\usetheme{default} 
\setbeamertemplate{navigation symbols}{} %no navigation symbols
\setbeamercolor{structure}{fg=\yourowntexcol} 
\setbeamercolor{normal text}{fg=\yourowntexcol} 



%%%%%%%%%%%%%%%%%%%%%%%%%
%%% CUSTOMISATIONS %%%%%%
%%%%%%%%%%%%%%%%%%%%%%%%%

% THE FOLLOWING COLOURS ARE PREDEFINED IN THE CLASS
%bi -- WHITE
%cz -- BLACK
%sz -- GRAY
%nieb -- BLUE
%ziel -- GREEN
%pom -- ORANGE
%% YOU CAN DEFINE YOUR OWN COLOUR TO USE HERE. SEE MAN.PDF


%%%% SLIDE ELEMENTS
\newcommand{\structureopacity}{0.75} %opacity for the structure elements (boxes and dots)
\newcommand{\strcolor}{ziel} %elements colour (predefined nieb; pom; ziel)

%%%% TEXT COLOUR
\newcommand{\yourowntexcol}{bi}



%%%%%%%%%%%%%%%%%%%%%%%%%
%%% TITLE SLIDE DATA %%%%
%%%%%%%%%%%%%%%%%%%%%%%%%
\newcommand{\titlephrase}{Mesh Processing with OpenMesh and C++}
\newcommand{\name}{Alexandre Kaspar}
\newcommand{\affil}{EPFL / MIT}
\newcommand{\email}{akaspar@mit.edu}





\begin{document}

%\fontencoding{T1}
%\fontfamily{serif}
%\fontseries{m}
%\fontshape{it}
%\fontsize{12}{15}
%\selectfont

\startingslide %this generates titlepage from the data above




\fbckg{figures/3dparis}
\begin{frame}
\pointedsl{C++}
\end{frame}

% based on http://ocw.mit.edu/courses/electrical-engineering-and-computer-science/6-096-introduction-to-c-january-iap-2011/index.htm

%\fbckg{figures/3dparis}
\begin{frame}
\framedsl{explained clearly}
\end{frame}


\fbckg{figures/3dparis}
\begin{frame}
\itemized{
\item BEAMER EASE OF USE
\item MODERN LOOK \& FEEL
}
\end{frame}






\fbckg{figures/3dparis}
\begin{frame}
\framedsl{
	\pitem{pointed slogan}
	\pitem{framed slogan}
	\pitem{beamer features}
}
\end{frame}







\fbckg{figures/3dparis}
\begin{frame}
  \thankyou   %%%% ending slide with thank you notice
\end{frame}





\fbckg{figures/3dparis}
\begin{frame}
\sources{
\includegraphics[width=1cm]{figures/C++-logo} \ flickr/lovelornpoets\\
\includegraphics[width=1cm]{figures/C++-logo} \ flickr/apsmuseum
}
\end{frame}


\end{document}